\documentclass[review]{elsarticle}

\usepackage{lineno,hyperref}
\usepackage[nolist]{acronym}
\usepackage[onehalfspacing]{setspace}
\usepackage{booktabs}
\usepackage{float}
\usepackage{amsmath}
\usepackage{listings}
\modulolinenumbers[1]

\journal{Astroparticle Physics}

\bibliographystyle{elsarticle-num}

\begin{document}

\begin{frontmatter}

\title{Recursive Single Photon Extraction\\ and Feature Generation on novel Photon Stream\\for Cherenkov Telescopes}
%
\author[b]{J.~Adam}
\author[a]{M.~L.~Ahnen}
\author[b]{D.~Baack}
\author[c]{M.~Balbo}
\author[d]{M.~Bergmann}
\author[a]{A.~Biland}
\author[d]{M.~Blank}
\author[a,1]{T.~Bretz}
\author[d]{K.~A.~Bruegge}
\author[b]{J.~Buss\corref{jbuss}}
\author[c]{A.~Dmytriiev}
\author[b]{M.~Domke}
\author[d,2]{D.~Dorner}
\author[b]{S.~Einecke}
\author[d]{C.~Hempfling}
\author[a]{D.~Hildebrand}
\author[a]{G.~Hughes}
\author[b]{L.~Linhoff}
\author[d]{K.~Mannheim}
\author[a]{S.~A.~Mueller\corref{smueller}}
\author[a]{D.~Neise}
\author[c]{A.~Neronov}
\author[b]{M.~Noethe}
\author[d]{A.~Paravac}
\author[a]{F.~Pauss}
\author[b]{W.~Rhode}
\author[a]{A.~Shukla}
\author[b]{F.~Temme}
\author[b]{J.~Thaele}
\author[c]{R.~Walter} 
%
\cortext[smueller]{Sebastian A. Mueller, sebmuell@phys.ethz.ch}
\cortext[jbuss]{Jens Buss, jens.buss@tu-dortmund.de}
%
\address[a]{ETH Zurich, Institute for Particle Physics\\
Otto-Stern-Weg 5, 8093 Zurich, Switzerland}
%
\address[b]{University of Geneva,  ISDC Data Center for Astrophysics\\
 Chemin d'Ecogia 16,  1290 Versoix,  Switzerland}
%
\address[c]{Universit\"at W\"urzburg, Institute for Theoretical Physics and Astrophysics\\
Emil-Fischer-Str. 31, 97074 W\"urzburg,  Germany}
%
\address[d]{TU Dortmund, Experimental Physics 5\\
Otto-Hahn-Str. 4, 44221 Dortmund, Germany}
%
%------------------------------------------------------------------------------
\begin{abstract}
%
\acfp{iact} record images of extensive air showers to probe the very high energetic gamma ray sky.
%
The development of pixels with single photon sensitivity and fast read out electronics even enabled the recording of air shower video sequences.
%
The temporal intensity development of the images became a powerful addition to reconstruct the primary particle properties from the air shower records.
%
However, current reconstruction methods are stuck in the concept of only one single photon amplitude and only one single arrival time for each pixel. 
%
In this contribution, we present a no compromise, recursive single photon extraction from the raw sensor responses which we use to create the novel photon stream concept where every photon is taken into account and treated equally.
%
We will not only show how the novel photon stream improves the air shower reconstruction and lowers the primary particle energy reconstruction threshold but also that a photon stream is the most natural and most compact output format for an \acs{iact}.
%
\end{abstract}
%------------------------------------------------------------------------------
\begin{keyword}
Single photon detection, silicon photo multiplier, G-APD  
\end{keyword}
%------------------------------------------------------------------------------
\end{frontmatter}
%\linenumbers
%------------------------------------------------------------------------------
\newcommand{\RadioCoeffCresponse}{r_\text{resp}}
\newcommand{\RadioCoeffTexp}{r_\text{expo}}
\newcommand{\RadioConstant}{r_\text{const}}
\newcommand{\NormalizedMirrorResponse}{R}
\newcommand{\StarIntensity}{s}
\newcommand{\MirrorReflectionIntensity}{m}
\newcommand{\RelativePointing}{\Theta}
\newcommand{\texp}{T_\text{expo}}
\newcommand{\iflux}{I_\text{pix}}
\newcommand{\cresponse}{C_\text{pix}}
\newcommand{\geom}{\alpha}
\newcommand{\FigCapLabSca}[4]{
    \begin{figure}[H]
        \begin{center}
            \includegraphics[width=#4\textwidth]{#1}
            \caption[]{#2}
            \label{#3}
        \end{center}
    \end{figure}
}
\newcommand{\FigCapLab}[3]{
    \FigCapLabSca{#1}{#2}{#3}{1.0}
}
\newcommand{\TwoFigsSideBySide}[2]{
    \begin{minipage}[t]{0.485\linewidth}
        \vspace{-0.5cm}
        \includegraphics[width=1\textwidth]{#1}
    \end{minipage}
    \hfill
    \begin{minipage}[t]{0.485\linewidth}
        \vspace{-0.5cm}
        \includegraphics[width=1\textwidth]{#2}
        \vspace{-1cm}
    \end{minipage}
}
\newcommand{\SideBySide}[2]{
    \newline
    \begin{minipage}[t]{0.485\linewidth}
        #1
        \end{minipage}
    %
    \hfill
    %
        \begin{minipage}[t]{0.485\linewidth}
        #2
    \end{minipage}\\
}
%------------------------------------------------------------------------------
\section{Introduction}
%
The large effective collection area of \acfp{iact} has opened the very high energetic gamma ray sky in the 1 Terra electron volt regime to astronomy.
%
During the night, the ground based \acsp{iact} record images using the Cherenkov light emitted in extensive air showers which were induced by cosmic rays and gamma rays.
%
To contribute to astronomy, the \acsp{iact} reconstruct the primary particle properties from the recorded images.
%
From the vast pool of diffuse cosmic ray induced air shower records, an \acs{iact} on a statistical basis extracts the few gamma ray induced air shower records using spatial and temporal features found in the records.
%
%------------------------------------------------------------------------------
\section{Current methods}

%
%------------------------------------------------------------------------------
\section{Single Photon Extraction Algorithm}
\label{SecExtractor}
%
We extract the arrival times of each photon. 
%
\FigCapLabSca{figures/example.png}{
	Example photon amplitude time line of a image pixel. 
}{FigExampleTimeline}{1.0}
%------------------------------------------------------------------------------
\section{Photon Stream Format}
\label{SecPhotonStream}
%------------------------------------------------------------------------------
\section{Results}
%
%------------------------------------------------------------------------------
\section{Conclusion}

%
%------------------------------------------------------------------------------
\section*{Acknowledgments}
%
The important contributions from ETH Zurich grants ETH-10.08-2 and ETH-27.12-1 as well as the funding by the German BMBF (Verbundforschung Astro- und Astroteilchenphysik) and HAP (Helmoltz Alliance for Astro- particle Physics) are gratefully acknowledged. 
%
We are thankful for the very valuable contributions from E. Lorenz, D. Renker and G. Viertel during the early phase of the project. 
%
We thank the Instituto de Astrofisica de Canarias allowing us to operate the telescope at the Observatorio del Roque de los Muchachos in La Palma, the Max-Planck-Institut für Physik for providing us with the mount of the former HEGRA CT3 telescope, and the MAGIC collaboration for their support.
%
%------------------------------------------------------------------------------
\section*{References}
\bibliography{references.bib}
%
%------------------------------------------------------------------------------
\begin{acronym}
    \acro{cog}[CoG]{Center of Gravity}
	\acro{sccan}[SCCAN]{Solar Concentrator Characterization At Night}
	\acro{psf}[PSF]{Point Spread Function}
	\acro{fact}[FACT]{First Geiger-mode Avalanche Cherenkov Telescope}
	\acro{veritas}[VERITAS]{Very Energetic Radiation Imaging Telescope Array System}
    \acro{iact}[IACT]{Imaging Atmospheric Cherenkov Telescope}
    \acro{namod}[NAMOD]{Normalized and Asynchronous Mirror Orientation Determination}
\end{acronym}
\end{document}